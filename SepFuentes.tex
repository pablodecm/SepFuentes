
% Default to the notebook output style

    


% Inherit from the specified cell style.




    
\documentclass{article}

    
    
    \usepackage{graphicx} % Used to insert images
    \usepackage{adjustbox} % Used to constrain images to a maximum size 
    \usepackage{color} % Allow colors to be defined
    \usepackage{enumerate} % Needed for markdown enumerations to work
    \usepackage{geometry} % Used to adjust the document margins
    \usepackage{amsmath} % Equations
    \usepackage{amssymb} % Equations
    \usepackage[mathletters]{ucs} % Extended unicode (utf-8) support
    \usepackage[utf8x]{inputenc} % Allow utf-8 characters in the tex document
    \usepackage{fancyvrb} % verbatim replacement that allows latex
    \usepackage{grffile} % extends the file name processing of package graphics 
                         % to support a larger range 
    % The hyperref package gives us a pdf with properly built
    % internal navigation ('pdf bookmarks' for the table of contents,
    % internal cross-reference links, web links for URLs, etc.)
    \usepackage{hyperref}
    \usepackage{longtable} % longtable support required by pandoc >1.10
    \usepackage{booktabs}  % table support for pandoc > 1.12.2
    

    
    
    \definecolor{orange}{cmyk}{0,0.4,0.8,0.2}
    \definecolor{darkorange}{rgb}{.71,0.21,0.01}
    \definecolor{darkgreen}{rgb}{.12,.54,.11}
    \definecolor{myteal}{rgb}{.26, .44, .56}
    \definecolor{gray}{gray}{0.45}
    \definecolor{lightgray}{gray}{.95}
    \definecolor{mediumgray}{gray}{.8}
    \definecolor{inputbackground}{rgb}{.95, .95, .85}
    \definecolor{outputbackground}{rgb}{.95, .95, .95}
    \definecolor{traceback}{rgb}{1, .95, .95}
    % ansi colors
    \definecolor{red}{rgb}{.6,0,0}
    \definecolor{green}{rgb}{0,.65,0}
    \definecolor{brown}{rgb}{0.6,0.6,0}
    \definecolor{blue}{rgb}{0,.145,.698}
    \definecolor{purple}{rgb}{.698,.145,.698}
    \definecolor{cyan}{rgb}{0,.698,.698}
    \definecolor{lightgray}{gray}{0.5}
    
    % bright ansi colors
    \definecolor{darkgray}{gray}{0.25}
    \definecolor{lightred}{rgb}{1.0,0.39,0.28}
    \definecolor{lightgreen}{rgb}{0.48,0.99,0.0}
    \definecolor{lightblue}{rgb}{0.53,0.81,0.92}
    \definecolor{lightpurple}{rgb}{0.87,0.63,0.87}
    \definecolor{lightcyan}{rgb}{0.5,1.0,0.83}
    
    % commands and environments needed by pandoc snippets
    % extracted from the output of `pandoc -s`
    \DefineVerbatimEnvironment{Highlighting}{Verbatim}{commandchars=\\\{\}}
    % Add ',fontsize=\small' for more characters per line
    \newenvironment{Shaded}{}{}
    \newcommand{\KeywordTok}[1]{\textcolor[rgb]{0.00,0.44,0.13}{\textbf{{#1}}}}
    \newcommand{\DataTypeTok}[1]{\textcolor[rgb]{0.56,0.13,0.00}{{#1}}}
    \newcommand{\DecValTok}[1]{\textcolor[rgb]{0.25,0.63,0.44}{{#1}}}
    \newcommand{\BaseNTok}[1]{\textcolor[rgb]{0.25,0.63,0.44}{{#1}}}
    \newcommand{\FloatTok}[1]{\textcolor[rgb]{0.25,0.63,0.44}{{#1}}}
    \newcommand{\CharTok}[1]{\textcolor[rgb]{0.25,0.44,0.63}{{#1}}}
    \newcommand{\StringTok}[1]{\textcolor[rgb]{0.25,0.44,0.63}{{#1}}}
    \newcommand{\CommentTok}[1]{\textcolor[rgb]{0.38,0.63,0.69}{\textit{{#1}}}}
    \newcommand{\OtherTok}[1]{\textcolor[rgb]{0.00,0.44,0.13}{{#1}}}
    \newcommand{\AlertTok}[1]{\textcolor[rgb]{1.00,0.00,0.00}{\textbf{{#1}}}}
    \newcommand{\FunctionTok}[1]{\textcolor[rgb]{0.02,0.16,0.49}{{#1}}}
    \newcommand{\RegionMarkerTok}[1]{{#1}}
    \newcommand{\ErrorTok}[1]{\textcolor[rgb]{1.00,0.00,0.00}{\textbf{{#1}}}}
    \newcommand{\NormalTok}[1]{{#1}}
    
    % Define a nice break command that doesn't care if a line doesn't already
    % exist.
    \def\br{\hspace*{\fill} \\* }
    % Math Jax compatability definitions
    \def\gt{>}
    \def\lt{<}
    % Document parameters
    
\title{Separación de imágenes con técnicas de PCA e ICA}

    
    
\author{Pablo de Castro}

    

    % Pygments definitions
    
\makeatletter
\def\PY@reset{\let\PY@it=\relax \let\PY@bf=\relax%
    \let\PY@ul=\relax \let\PY@tc=\relax%
    \let\PY@bc=\relax \let\PY@ff=\relax}
\def\PY@tok#1{\csname PY@tok@#1\endcsname}
\def\PY@toks#1+{\ifx\relax#1\empty\else%
    \PY@tok{#1}\expandafter\PY@toks\fi}
\def\PY@do#1{\PY@bc{\PY@tc{\PY@ul{%
    \PY@it{\PY@bf{\PY@ff{#1}}}}}}}
\def\PY#1#2{\PY@reset\PY@toks#1+\relax+\PY@do{#2}}

\expandafter\def\csname PY@tok@gd\endcsname{\def\PY@tc##1{\textcolor[rgb]{0.63,0.00,0.00}{##1}}}
\expandafter\def\csname PY@tok@gu\endcsname{\let\PY@bf=\textbf\def\PY@tc##1{\textcolor[rgb]{0.50,0.00,0.50}{##1}}}
\expandafter\def\csname PY@tok@gt\endcsname{\def\PY@tc##1{\textcolor[rgb]{0.00,0.27,0.87}{##1}}}
\expandafter\def\csname PY@tok@gs\endcsname{\let\PY@bf=\textbf}
\expandafter\def\csname PY@tok@gr\endcsname{\def\PY@tc##1{\textcolor[rgb]{1.00,0.00,0.00}{##1}}}
\expandafter\def\csname PY@tok@cm\endcsname{\let\PY@it=\textit\def\PY@tc##1{\textcolor[rgb]{0.25,0.50,0.50}{##1}}}
\expandafter\def\csname PY@tok@vg\endcsname{\def\PY@tc##1{\textcolor[rgb]{0.10,0.09,0.49}{##1}}}
\expandafter\def\csname PY@tok@m\endcsname{\def\PY@tc##1{\textcolor[rgb]{0.40,0.40,0.40}{##1}}}
\expandafter\def\csname PY@tok@mh\endcsname{\def\PY@tc##1{\textcolor[rgb]{0.40,0.40,0.40}{##1}}}
\expandafter\def\csname PY@tok@go\endcsname{\def\PY@tc##1{\textcolor[rgb]{0.53,0.53,0.53}{##1}}}
\expandafter\def\csname PY@tok@ge\endcsname{\let\PY@it=\textit}
\expandafter\def\csname PY@tok@vc\endcsname{\def\PY@tc##1{\textcolor[rgb]{0.10,0.09,0.49}{##1}}}
\expandafter\def\csname PY@tok@il\endcsname{\def\PY@tc##1{\textcolor[rgb]{0.40,0.40,0.40}{##1}}}
\expandafter\def\csname PY@tok@cs\endcsname{\let\PY@it=\textit\def\PY@tc##1{\textcolor[rgb]{0.25,0.50,0.50}{##1}}}
\expandafter\def\csname PY@tok@cp\endcsname{\def\PY@tc##1{\textcolor[rgb]{0.74,0.48,0.00}{##1}}}
\expandafter\def\csname PY@tok@gi\endcsname{\def\PY@tc##1{\textcolor[rgb]{0.00,0.63,0.00}{##1}}}
\expandafter\def\csname PY@tok@gh\endcsname{\let\PY@bf=\textbf\def\PY@tc##1{\textcolor[rgb]{0.00,0.00,0.50}{##1}}}
\expandafter\def\csname PY@tok@ni\endcsname{\let\PY@bf=\textbf\def\PY@tc##1{\textcolor[rgb]{0.60,0.60,0.60}{##1}}}
\expandafter\def\csname PY@tok@nl\endcsname{\def\PY@tc##1{\textcolor[rgb]{0.63,0.63,0.00}{##1}}}
\expandafter\def\csname PY@tok@nn\endcsname{\let\PY@bf=\textbf\def\PY@tc##1{\textcolor[rgb]{0.00,0.00,1.00}{##1}}}
\expandafter\def\csname PY@tok@no\endcsname{\def\PY@tc##1{\textcolor[rgb]{0.53,0.00,0.00}{##1}}}
\expandafter\def\csname PY@tok@na\endcsname{\def\PY@tc##1{\textcolor[rgb]{0.49,0.56,0.16}{##1}}}
\expandafter\def\csname PY@tok@nb\endcsname{\def\PY@tc##1{\textcolor[rgb]{0.00,0.50,0.00}{##1}}}
\expandafter\def\csname PY@tok@nc\endcsname{\let\PY@bf=\textbf\def\PY@tc##1{\textcolor[rgb]{0.00,0.00,1.00}{##1}}}
\expandafter\def\csname PY@tok@nd\endcsname{\def\PY@tc##1{\textcolor[rgb]{0.67,0.13,1.00}{##1}}}
\expandafter\def\csname PY@tok@ne\endcsname{\let\PY@bf=\textbf\def\PY@tc##1{\textcolor[rgb]{0.82,0.25,0.23}{##1}}}
\expandafter\def\csname PY@tok@nf\endcsname{\def\PY@tc##1{\textcolor[rgb]{0.00,0.00,1.00}{##1}}}
\expandafter\def\csname PY@tok@si\endcsname{\let\PY@bf=\textbf\def\PY@tc##1{\textcolor[rgb]{0.73,0.40,0.53}{##1}}}
\expandafter\def\csname PY@tok@s2\endcsname{\def\PY@tc##1{\textcolor[rgb]{0.73,0.13,0.13}{##1}}}
\expandafter\def\csname PY@tok@vi\endcsname{\def\PY@tc##1{\textcolor[rgb]{0.10,0.09,0.49}{##1}}}
\expandafter\def\csname PY@tok@nt\endcsname{\let\PY@bf=\textbf\def\PY@tc##1{\textcolor[rgb]{0.00,0.50,0.00}{##1}}}
\expandafter\def\csname PY@tok@nv\endcsname{\def\PY@tc##1{\textcolor[rgb]{0.10,0.09,0.49}{##1}}}
\expandafter\def\csname PY@tok@s1\endcsname{\def\PY@tc##1{\textcolor[rgb]{0.73,0.13,0.13}{##1}}}
\expandafter\def\csname PY@tok@sh\endcsname{\def\PY@tc##1{\textcolor[rgb]{0.73,0.13,0.13}{##1}}}
\expandafter\def\csname PY@tok@sc\endcsname{\def\PY@tc##1{\textcolor[rgb]{0.73,0.13,0.13}{##1}}}
\expandafter\def\csname PY@tok@sx\endcsname{\def\PY@tc##1{\textcolor[rgb]{0.00,0.50,0.00}{##1}}}
\expandafter\def\csname PY@tok@bp\endcsname{\def\PY@tc##1{\textcolor[rgb]{0.00,0.50,0.00}{##1}}}
\expandafter\def\csname PY@tok@c1\endcsname{\let\PY@it=\textit\def\PY@tc##1{\textcolor[rgb]{0.25,0.50,0.50}{##1}}}
\expandafter\def\csname PY@tok@kc\endcsname{\let\PY@bf=\textbf\def\PY@tc##1{\textcolor[rgb]{0.00,0.50,0.00}{##1}}}
\expandafter\def\csname PY@tok@c\endcsname{\let\PY@it=\textit\def\PY@tc##1{\textcolor[rgb]{0.25,0.50,0.50}{##1}}}
\expandafter\def\csname PY@tok@mf\endcsname{\def\PY@tc##1{\textcolor[rgb]{0.40,0.40,0.40}{##1}}}
\expandafter\def\csname PY@tok@err\endcsname{\def\PY@bc##1{\setlength{\fboxsep}{0pt}\fcolorbox[rgb]{1.00,0.00,0.00}{1,1,1}{\strut ##1}}}
\expandafter\def\csname PY@tok@kd\endcsname{\let\PY@bf=\textbf\def\PY@tc##1{\textcolor[rgb]{0.00,0.50,0.00}{##1}}}
\expandafter\def\csname PY@tok@ss\endcsname{\def\PY@tc##1{\textcolor[rgb]{0.10,0.09,0.49}{##1}}}
\expandafter\def\csname PY@tok@sr\endcsname{\def\PY@tc##1{\textcolor[rgb]{0.73,0.40,0.53}{##1}}}
\expandafter\def\csname PY@tok@mo\endcsname{\def\PY@tc##1{\textcolor[rgb]{0.40,0.40,0.40}{##1}}}
\expandafter\def\csname PY@tok@kn\endcsname{\let\PY@bf=\textbf\def\PY@tc##1{\textcolor[rgb]{0.00,0.50,0.00}{##1}}}
\expandafter\def\csname PY@tok@mi\endcsname{\def\PY@tc##1{\textcolor[rgb]{0.40,0.40,0.40}{##1}}}
\expandafter\def\csname PY@tok@gp\endcsname{\let\PY@bf=\textbf\def\PY@tc##1{\textcolor[rgb]{0.00,0.00,0.50}{##1}}}
\expandafter\def\csname PY@tok@o\endcsname{\def\PY@tc##1{\textcolor[rgb]{0.40,0.40,0.40}{##1}}}
\expandafter\def\csname PY@tok@kr\endcsname{\let\PY@bf=\textbf\def\PY@tc##1{\textcolor[rgb]{0.00,0.50,0.00}{##1}}}
\expandafter\def\csname PY@tok@s\endcsname{\def\PY@tc##1{\textcolor[rgb]{0.73,0.13,0.13}{##1}}}
\expandafter\def\csname PY@tok@kp\endcsname{\def\PY@tc##1{\textcolor[rgb]{0.00,0.50,0.00}{##1}}}
\expandafter\def\csname PY@tok@w\endcsname{\def\PY@tc##1{\textcolor[rgb]{0.73,0.73,0.73}{##1}}}
\expandafter\def\csname PY@tok@kt\endcsname{\def\PY@tc##1{\textcolor[rgb]{0.69,0.00,0.25}{##1}}}
\expandafter\def\csname PY@tok@ow\endcsname{\let\PY@bf=\textbf\def\PY@tc##1{\textcolor[rgb]{0.67,0.13,1.00}{##1}}}
\expandafter\def\csname PY@tok@sb\endcsname{\def\PY@tc##1{\textcolor[rgb]{0.73,0.13,0.13}{##1}}}
\expandafter\def\csname PY@tok@k\endcsname{\let\PY@bf=\textbf\def\PY@tc##1{\textcolor[rgb]{0.00,0.50,0.00}{##1}}}
\expandafter\def\csname PY@tok@se\endcsname{\let\PY@bf=\textbf\def\PY@tc##1{\textcolor[rgb]{0.73,0.40,0.13}{##1}}}
\expandafter\def\csname PY@tok@sd\endcsname{\let\PY@it=\textit\def\PY@tc##1{\textcolor[rgb]{0.73,0.13,0.13}{##1}}}

\def\PYZbs{\char`\\}
\def\PYZus{\char`\_}
\def\PYZob{\char`\{}
\def\PYZcb{\char`\}}
\def\PYZca{\char`\^}
\def\PYZam{\char`\&}
\def\PYZlt{\char`\<}
\def\PYZgt{\char`\>}
\def\PYZsh{\char`\#}
\def\PYZpc{\char`\%}
\def\PYZdl{\char`\$}
\def\PYZhy{\char`\-}
\def\PYZsq{\char`\'}
\def\PYZdq{\char`\"}
\def\PYZti{\char`\~}
% for compatibility with earlier versions
\def\PYZat{@}
\def\PYZlb{[}
\def\PYZrb{]}
\makeatother


    % Exact colors from NB
    \definecolor{incolor}{rgb}{0.0, 0.0, 0.5}
    \definecolor{outcolor}{rgb}{0.545, 0.0, 0.0}



    
    % Prevent overflowing lines due to hard-to-break entities
    \sloppy 
    % Setup hyperref package
    \hypersetup{
      breaklinks=true,  % so long urls are correctly broken across lines
      colorlinks=true,
      urlcolor=blue,
      linkcolor=darkorange,
      citecolor=darkgreen,
      }
    % Slightly bigger margins than the latex defaults
    
    \geometry{verbose,tmargin=1in,bmargin=1in,lmargin=1in,rmargin=1in}
    
    

    \begin{document}
    
    
    \maketitle
    
    

    

    \section{Introducción}


    El objetivo de esta práctica es la aplicación de las técnicas de
Análisis por Componentes Principales (\emph{PCA, del inglés}) y de
Análisis por Componentes Independientes (\emph{ICA, del inglés}), para
separas tres imágenes que se encuetran mezcladas para extraer las
componentes y la matríz de mezcla.


    \section{Fundamento Teórico}



    \subsection{Problema Propuesto}


    Se han provisto tres imágenes de \(512\times512\) píxeles cada una
conteniendo una mezcla lineal de tres imágenes originales. Como se
mostrará en la siguiente sección, dos de las imágenes originales pueden
distinguirse claramente (una mujer joven y un gato). Sin embargo, la
tercera imágen no es claramente identificable por lo que está oculta.

Las imágenes a utilizar no contienen ningún componente de ruido. El
modelo de mezcla utilizado se puede describir como:

\[ \mathbf{Y} = \mathbf{A} \mathbf{X} \]

en donde \(\mathbf{Y}\) son las tres imágenes mezcladas (dimensión
\(3\times512^2\)), \(\mathbf{X}\) las tres imágenes originales
(dimensión \(3\times512^2\)) y \(\mathbf{A}\) es la matrix de mezcla
(dimensión \(3\times3\)). Se desconoce tanto la matriz de mezcla
\(\mathbf{A}\) como las componentes originales \(\mathbf{X}\). Este
problema, en general, carece de solución determinada. Sin embargo, en
este trabajo se utilizaran las técnicas de Análisis de Componentes
Principales y Análisis por Componentes Independientes para estimar ambos
elementos bajo ciertas condiciones impuestas. La aplicación de estos
métodos también tiene como objetivo la identificación de la tercera
imagen original que se encuentra oculta.


    \subsection{Análisis de Componentes Principales}


    El Análisis de Componentes Principales (PCA, del inglés), es un
prodedimiento estadístico para transformar un conjunto de observaciones
mediante una transformación ortogonal a otro conjunto de componentes que
esten descorrelacionadas, denominadas componentes principales.

Esta tranformación se define de tal modo que cada componente tenga la
máxima varianza posible bajo la condición de que la transfromación se
ortogonal a las componentes anteriores. El procedimiento permite, de
forma no paramétrica, la compresión o representación en un sistema de
coordenadas que explique mejor la varianza de los datos.

Supóngase que se tiene una matríz de datos \(\mathbf{Y}\) (con
dimensiones \((\rm{nº~componentes}) \times (\rm{nº~muestras}\))), el
objetivo es encontrar un tranformación lineal ortonormal \(\mathbf{P}\)
que genere una nueva representación de los datos \(\mathbf{Z}\) en los
que las componentes esten descorrelacionadas:

\[ \mathbf{Z} = \mathbf{P} \mathbf{Y} \]

La correlación entre las componentes se puede evaluar con el valor de
los elementos no diagonales de la matriz de covarianza \(\mathbf{C_Y}\),
que para la matríz de datos \(\mathbf{Y}\) una vez centralizada
(restando la media de cada componente) se define (sin tener en cuenta
los factores constantes) como:

\[ \mathbf{C_Y} = \mathbf{Y} \mathbf{Y^T} \]

El objetivo es que en la representación \(\mathbf{Z}\) las componentes
esten descorrelacionadas, es decir que los elementos no diagonales de la
matríz sean nulos \cite{jolliffe2005principal}. La matrix de covarianza
\(\mathbf{C_Z}\) para matríz de datos \(\mathbf{Z}\) se define como:

\[ \mathbf{C_Z} = \mathbf{Z} \mathbf{Z^T} = \mathbf{P} \mathbf{Y} (\mathbf{P} \mathbf{Y})^T =
\mathbf{P} \mathbf{Y} \mathbf{Y^T} \mathbf{P}^T = \mathbf{P} \mathbf{C_Y} \mathbf{P}^T \]

Dado que la matríz \(\mathbf{P}\) es ortogonal, su tranversa es igual a
su inversa \(\mathbf{P^T}=\mathbf{P^{-1}}\), por lo que \(\mathbf{P}\)
es la matríz que diagonaliza la matríz de covarianza de los datos
\(\mathbf{Y}\). En resumen, el problema de encontrar los componentes
principales se reduce a la búsqueda de los autovectores de
\(\mathbf{C_Y}\), para lo cuál se pueden utiliza diversos métodos
(i.e.~EVD y SVD) . La primera componente correspondera con el autovector
del autovalor mayor en valor absoluto, la segunda componente al segundo
mayor y así sucesivamente.

Este método es útil para la compresión y representación de datos, ya que
permite cambiar a una base en la que las componentes estén
descorrelacionadas mediante una tranformación lineal. Por lo tanto,
cuando se aplique sobre los datos de este problema se espera que cada
componente principal corresponda aproximadamente a una imágen inicial.


    \subsection{Análisis de Componentes Independientes}


    El Análisis por Componentes Independientes (\emph{ICA, del inglés}) es
un método computacional para separar componentes estadísticamente
independientes presentes en un conjunto de datos. Este proceso también
se conoce como separación de fuentes independientes y es una posible
solución para el problema propuesto en esta práctica.

Para la aplicación este procedimiento, se supone que las fuentes
(i.e.~componentes de \(\mathbf{X}\)) son estadísticamente
independientes, que tienen distribuciones no gaussianas (aunque una sola
componente si que puede ser gaussiana) y que a matriz de mezcla
\(\mathbf{A}\) es invertible. Bajo dichas condiciones el sistema es
identificable y sus componentes independientes pueden determinarse salvo
permutaciones y factores multiplicativos.

Existen diversos métodos para la obtención de las componentes
principales, incluyendo la maximización de la no-gaussianidad
(i.e.~kurtosis o negentropía), máxima verosimilitud, mínima información
nula, métodos tensoriales y de decorrelación no lineal. Muchos de los
métodos mencionados requiren la aplicación de PCA como un primer paso
para la preparación de los datos, que se denomina blanqueado
(\emph{withening}). El objetivo último de todos los métodos es la
busqueda de una matríz de componentes \(\mathbf{W}\), que es la inversa
de la matriz de mezcla \(\mathbf{A}\), de modo que las componentes de
\(\mathbf{X}\) sean estadísticamente independientes.


    \section{Desarrollo}



    \subsection{Software Utilizado}


    El análisis de las imágenes en esta práctica se ha realizado en el
entorno de computación interactivo \emph{IPython Notebook}
\cite{PER-GRA:2007}. Este mismo documento puede ser visualizado o
descargado en formato digital en
http://nbviewer.ipython.org/github/pablodecm/SepFuentes/blob/master/SepFuentes.ipynb.

Se han utilizado las bibliotecas de software \emph{open source}:
\emph{numpy} \cite{725236} para la manipulación de arrays y matrices,
\emph{matplotlib} \cite{Hunter:2007} para la visualización de imágenes,
\emph{scikit-learn} \cite{scikit-learn} para las implementaciones de PCA
e ICA utilizadas y \emph{sympy} \cite{sympy} para el formato de la
salida de datos de matrices.

    A continuación, se importan todos los modulos necesarios de la
bibliotecas que serán de utilidad en los apartados posteriores.

    \begin{Verbatim}[commandchars=\\\{\}]
{\color{incolor}In [{\color{incolor}1}]:} \PY{o}{\PYZpc{}}\PY{k}{matplotlib} \PY{n}{inline}
        \PY{k+kn}{import} \PY{n+nn}{numpy} \PY{k+kn}{as} \PY{n+nn}{np}
        \PY{k+kn}{import} \PY{n+nn}{matplotlib.pyplot} \PY{k+kn}{as} \PY{n+nn}{plt}
        \PY{k+kn}{from} \PY{n+nn}{sklearn} \PY{k+kn}{import} \PY{n}{decomposition}
        \PY{k+kn}{import} \PY{n+nn}{sympy}
        \PY{n}{sympy}\PY{o}{.}\PY{n}{init\PYZus{}printing}\PY{p}{(}\PY{n}{use\PYZus{}latex}\PY{o}{=}\PY{l+s}{\PYZsq{}}\PY{l+s}{mathjax}\PY{l+s}{\PYZsq{}}\PY{p}{)}
\end{Verbatim}


    \subsection{Importación y Visualización de las Imágenes}


    Las imágenes mezcladas se encuentran en los archivos de texto:

\begin{itemize}
\itemsep1pt\parskip0pt\parsep0pt
\item
  imagen\_mezclada\_uno.dat
\item
  imagen\_mexclada\_dos.dat
\item
  imagen\_mexclada\_tres.dat
\end{itemize}

En este apartado, se van a importar las imágenes a un formato de array
multidimensional y se van a visualizar:

    \begin{Verbatim}[commandchars=\\\{\}]
{\color{incolor}In [{\color{incolor}2}]:} \PY{n}{img\PYZus{}1} \PY{o}{=} \PY{n}{np}\PY{o}{.}\PY{n}{genfromtxt}\PY{p}{(}\PY{l+s}{\PYZsq{}}\PY{l+s}{imagen\PYZus{}mezclada\PYZus{}uno.dat}\PY{l+s}{\PYZsq{}}\PY{p}{,} \PY{n}{dtype}\PY{o}{=}\PY{l+s}{\PYZsq{}}\PY{l+s}{float64}\PY{l+s}{\PYZsq{}}\PY{p}{)}
        \PY{n}{img\PYZus{}2} \PY{o}{=} \PY{n}{np}\PY{o}{.}\PY{n}{genfromtxt}\PY{p}{(}\PY{l+s}{\PYZsq{}}\PY{l+s}{imagen\PYZus{}mexclada\PYZus{}dos.dat}\PY{l+s}{\PYZsq{}}\PY{p}{,} \PY{n}{dtype}\PY{o}{=}\PY{l+s}{\PYZsq{}}\PY{l+s}{float64}\PY{l+s}{\PYZsq{}}\PY{p}{)}
        \PY{n}{img\PYZus{}3} \PY{o}{=} \PY{n}{np}\PY{o}{.}\PY{n}{genfromtxt}\PY{p}{(}\PY{l+s}{\PYZsq{}}\PY{l+s}{imagen\PYZus{}mexclada\PYZus{}tres.dat}\PY{l+s}{\PYZsq{}}\PY{p}{,} \PY{n}{dtype}\PY{o}{=}\PY{l+s}{\PYZsq{}}\PY{l+s}{float64}\PY{l+s}{\PYZsq{}}\PY{p}{)}
        
        \PY{n}{f}\PY{p}{,} \PY{p}{(}\PY{n}{ax1}\PY{p}{,}\PY{n}{ax2}\PY{p}{,}\PY{n}{ax3}\PY{p}{)} \PY{o}{=} \PY{n}{plt}\PY{o}{.}\PY{n}{subplots}\PY{p}{(}\PY{l+m+mi}{1}\PY{p}{,}\PY{l+m+mi}{3}\PY{p}{)}
        \PY{n}{f}\PY{o}{.}\PY{n}{set\PYZus{}size\PYZus{}inches}\PY{p}{(}\PY{p}{(}\PY{l+m+mi}{24}\PY{p}{,}\PY{l+m+mi}{40}\PY{p}{)}\PY{p}{)}
        \PY{n}{ax1}\PY{o}{.}\PY{n}{imshow}\PY{p}{(}\PY{n}{img\PYZus{}1}\PY{p}{,} \PY{n}{cmap}\PY{o}{=}\PY{n}{plt}\PY{o}{.}\PY{n}{cm}\PY{o}{.}\PY{n}{gray}\PY{p}{)}
        \PY{n}{ax1}\PY{o}{.}\PY{n}{set\PYZus{}title}\PY{p}{(}\PY{l+s}{\PYZdq{}}\PY{l+s}{Imagen Mezclada 1}\PY{l+s}{\PYZdq{}}\PY{p}{)}
        \PY{n}{ax2}\PY{o}{.}\PY{n}{imshow}\PY{p}{(}\PY{n}{img\PYZus{}2}\PY{p}{,} \PY{n}{cmap}\PY{o}{=}\PY{n}{plt}\PY{o}{.}\PY{n}{cm}\PY{o}{.}\PY{n}{gray}\PY{p}{)}
        \PY{n}{ax2}\PY{o}{.}\PY{n}{set\PYZus{}title}\PY{p}{(}\PY{l+s}{\PYZdq{}}\PY{l+s}{Imagen Mezclada 2}\PY{l+s}{\PYZdq{}}\PY{p}{)}
        \PY{n}{ax3}\PY{o}{.}\PY{n}{imshow}\PY{p}{(}\PY{n}{img\PYZus{}3}\PY{p}{,} \PY{n}{cmap}\PY{o}{=}\PY{n}{plt}\PY{o}{.}\PY{n}{cm}\PY{o}{.}\PY{n}{gray}\PY{p}{)}
        \PY{n}{ax3}\PY{o}{.}\PY{n}{set\PYZus{}title}\PY{p}{(}\PY{l+s}{\PYZdq{}}\PY{l+s}{Imagen Mezclada 3}\PY{l+s}{\PYZdq{}}\PY{p}{)}
\end{Verbatim}

            \begin{Verbatim}[commandchars=\\\{\}]
{\color{outcolor}Out[{\color{outcolor}2}]:} <matplotlib.text.Text at 0x1089bea10>
\end{Verbatim}
        
    \begin{center}
    \adjustimage{max size={0.9\linewidth}{0.9\paperheight}}{SepFuentes_files/SepFuentes_16_1.png}
    \end{center}
    { \hspace*{\fill} \\}
    
    Se aprecia que las imagenes provistas contienen tres imágenes mezcladas.
La visualización permite distinguir visualmente tres objetos
identificables a una mujer jover, un gato y lo que parece una página de
un libro con texto e ilustraciones.

Se han importado y visualizado las imágenes, sin embargo, el formato de
matriz no es adecuado para la aplicación de PCA e ICA. Por dicho motivo,
se van a transformar la matrices \(512\times 512\) en vectores de 262144
elementos. Las tres componentes mezcladas a su vez se apilan
verticalmente para formar la matríz \(\mathbf{Y}\), de dimensiones
\(3\times 262144\).

    \begin{Verbatim}[commandchars=\\\{\}]
{\color{incolor}In [{\color{incolor}3}]:} \PY{c}{\PYZsh{} each image is flattened and all three are stacked together}
        \PY{n}{Y} \PY{o}{=} \PY{n}{np}\PY{o}{.}\PY{n}{vstack}\PY{p}{(}\PY{p}{(}\PY{n}{img\PYZus{}1}\PY{o}{.}\PY{n}{flatten}\PY{p}{(}\PY{p}{)}\PY{p}{,}\PY{n}{img\PYZus{}2}\PY{o}{.}\PY{n}{flatten}\PY{p}{(}\PY{p}{)}\PY{p}{,}\PY{n}{img\PYZus{}3}\PY{o}{.}\PY{n}{flatten}\PY{p}{(}\PY{p}{)}\PY{p}{)}\PY{p}{)}
        \PY{n}{Y}\PY{o}{.}\PY{n}{shape}
\end{Verbatim}
\texttt{\color{outcolor}Out[{\color{outcolor}3}]:}
    
    
        \begin{equation*}
        \begin{pmatrix}3, & 262144\end{pmatrix}
        \end{equation*}

    


    \subsection{Aplicación del método PCA}


    La impleméntacion del método de componenetes principales utilizada es la
provista por el módulo \emph{decomposition.PCA} de \emph{scikit-learn}.
Se basa en la el método de descomposición en valores singulares (SVD en
inglés).

A continuación, se aplica el método a la matríz \(\mathbf{Y}\) y se
visualizan las componentes principales de los datos de nuevo en forma de
imagen:

    \begin{Verbatim}[commandchars=\\\{\}]
{\color{incolor}In [{\color{incolor}4}]:} \PY{n}{pca} \PY{o}{=} \PY{n}{decomposition}\PY{o}{.}\PY{n}{PCA}\PY{p}{(}\PY{n}{n\PYZus{}components}\PY{o}{=}\PY{l+m+mi}{3}\PY{p}{)}
        \PY{n}{img\PYZus{}pca}\PY{o}{=} \PY{n}{pca}\PY{o}{.}\PY{n}{fit\PYZus{}transform}\PY{p}{(}\PY{n}{Y}\PY{o}{.}\PY{n}{T}\PY{p}{)}
        
        \PY{n}{f}\PY{p}{,} \PY{p}{(}\PY{n}{ax1}\PY{p}{,}\PY{n}{ax2}\PY{p}{,}\PY{n}{ax3}\PY{p}{)} \PY{o}{=} \PY{n}{plt}\PY{o}{.}\PY{n}{subplots}\PY{p}{(}\PY{l+m+mi}{1}\PY{p}{,}\PY{l+m+mi}{3}\PY{p}{)}
        \PY{n}{f}\PY{o}{.}\PY{n}{set\PYZus{}size\PYZus{}inches}\PY{p}{(}\PY{p}{(}\PY{l+m+mi}{24}\PY{p}{,}\PY{l+m+mi}{40}\PY{p}{)}\PY{p}{)}
        \PY{n}{ax1}\PY{o}{.}\PY{n}{imshow}\PY{p}{(}\PY{n}{img\PYZus{}pca}\PY{p}{[}\PY{p}{:}\PY{p}{,}\PY{l+m+mi}{0}\PY{p}{]}\PY{o}{.}\PY{n}{reshape}\PY{p}{(}\PY{p}{(}\PY{l+m+mi}{512}\PY{p}{,}\PY{l+m+mi}{512}\PY{p}{)}\PY{p}{)}\PY{p}{,} \PY{n}{cmap}\PY{o}{=}\PY{n}{plt}\PY{o}{.}\PY{n}{cm}\PY{o}{.}\PY{n}{gray}\PY{p}{)}
        \PY{n}{ax1}\PY{o}{.}\PY{n}{set\PYZus{}title}\PY{p}{(}\PY{l+s}{\PYZdq{}}\PY{l+s}{Componente Principal 1}\PY{l+s}{\PYZdq{}}\PY{p}{)}
        \PY{n}{ax2}\PY{o}{.}\PY{n}{imshow}\PY{p}{(}\PY{n}{img\PYZus{}pca}\PY{p}{[}\PY{p}{:}\PY{p}{,}\PY{l+m+mi}{1}\PY{p}{]}\PY{o}{.}\PY{n}{reshape}\PY{p}{(}\PY{p}{(}\PY{l+m+mi}{512}\PY{p}{,}\PY{l+m+mi}{512}\PY{p}{)}\PY{p}{)}\PY{p}{,} \PY{n}{cmap}\PY{o}{=}\PY{n}{plt}\PY{o}{.}\PY{n}{cm}\PY{o}{.}\PY{n}{gray}\PY{p}{)}
        \PY{n}{ax2}\PY{o}{.}\PY{n}{set\PYZus{}title}\PY{p}{(}\PY{l+s}{\PYZdq{}}\PY{l+s}{Componente Principal 2}\PY{l+s}{\PYZdq{}}\PY{p}{)}
        \PY{n}{ax3}\PY{o}{.}\PY{n}{imshow}\PY{p}{(}\PY{n}{img\PYZus{}pca}\PY{p}{[}\PY{p}{:}\PY{p}{,}\PY{l+m+mi}{2}\PY{p}{]}\PY{o}{.}\PY{n}{reshape}\PY{p}{(}\PY{p}{(}\PY{l+m+mi}{512}\PY{p}{,}\PY{l+m+mi}{512}\PY{p}{)}\PY{p}{)}\PY{p}{,} \PY{n}{cmap}\PY{o}{=}\PY{n}{plt}\PY{o}{.}\PY{n}{cm}\PY{o}{.}\PY{n}{gray}\PY{p}{)}
        \PY{n}{ax3}\PY{o}{.}\PY{n}{set\PYZus{}title}\PY{p}{(}\PY{l+s}{\PYZdq{}}\PY{l+s}{Componente Principal 3}\PY{l+s}{\PYZdq{}}\PY{p}{)}
\end{Verbatim}

            \begin{Verbatim}[commandchars=\\\{\}]
{\color{outcolor}Out[{\color{outcolor}4}]:} <matplotlib.text.Text at 0x108c19f90>
\end{Verbatim}
        
    \begin{center}
    \adjustimage{max size={0.9\linewidth}{0.9\paperheight}}{SepFuentes_files/SepFuentes_21_1.png}
    \end{center}
    { \hspace*{\fill} \\}
    
    Se observa que las componentes obtenidas son más facilmente
distinguibles que en el caso anterior. Sin embargo, como se esperaba, el
PCA no ha conseguido separar las imágenes en su totalidad. Esto se
aprecia claramente para la tercera componente que es una combinación de
la imagen de mujer joven y de la imagen del gato. Es posible también
obtener la varianza (autovalores)explicada por cada componente en
porcentaje:

    \begin{Verbatim}[commandchars=\\\{\}]
{\color{incolor}In [{\color{incolor}5}]:} \PY{n}{sympy}\PY{o}{.}\PY{n}{Matrix}\PY{p}{(}\PY{n}{pca}\PY{o}{.}\PY{n}{explained\PYZus{}variance\PYZus{}ratio\PYZus{}}\PY{o}{.}\PY{n}{round}\PY{p}{(}\PY{n}{decimals}\PY{o}{=}\PY{l+m+mi}{4}\PY{p}{)}\PY{p}{)}
\end{Verbatim}
\texttt{\color{outcolor}Out[{\color{outcolor}5}]:}
    
    
        \begin{equation*}
        \left[\begin{matrix}0.8389 & 0.1334 & 0.0278\end{matrix}\right]
        \end{equation*}

    

    La primera componenente explica más del \(83 \%\) de la varianza total,
mientras que los porcentajes de las otras dos son mucho más pequeños. Se
puede obtener la matríz \(\mathbf{P}\) con los autovectores de la matríz
de covarianza.

    \begin{Verbatim}[commandchars=\\\{\}]
{\color{incolor}In [{\color{incolor}6}]:} \PY{n}{sympy}\PY{o}{.}\PY{n}{Eq}\PY{p}{(} \PY{n}{sympy}\PY{o}{.}\PY{n}{MatrixSymbol}\PY{p}{(}\PY{l+s}{\PYZdq{}}\PY{l+s}{\PYZbs{}}\PY{l+s}{mathbf\PYZob{}P\PYZcb{}}\PY{l+s}{\PYZdq{}}\PY{p}{,}\PY{l+m+mi}{3}\PY{p}{,}\PY{l+m+mi}{3}\PY{p}{)}\PY{p}{,}
                 \PY{n}{sympy}\PY{o}{.}\PY{n}{Matrix}\PY{p}{(}\PY{n}{pca}\PY{o}{.}\PY{n}{components\PYZus{}}\PY{o}{.}\PY{n}{round}\PY{p}{(}\PY{n}{decimals}\PY{o}{=}\PY{l+m+mi}{4}\PY{p}{)}\PY{p}{)}\PY{p}{)}
\end{Verbatim}
\texttt{\color{outcolor}Out[{\color{outcolor}6}]:}
    
    
        \begin{equation*}
        \mathbf{P} = \left[\begin{matrix}0.6591 & 0.1426 & 0.7384\\0.7432 & -0.2742 & -0.6104\\0.1154 & 0.951 & -0.2867\end{matrix}\right]
        \end{equation*}

    

    Una primera estimación de la matríz de mezcla \(\mathbf{A}_{PCA}\) puede
ser obtenida como la inversa de la tranformación \(\mathbf{P}\)
obteniendo:

    \begin{Verbatim}[commandchars=\\\{\}]
{\color{incolor}In [{\color{incolor}7}]:} \PY{n}{A\PYZus{}pca} \PY{o}{=} \PY{n}{np}\PY{o}{.}\PY{n}{linalg}\PY{o}{.}\PY{n}{inv}\PY{p}{(}\PY{n}{pca}\PY{o}{.}\PY{n}{components\PYZus{}}\PY{p}{)}
        \PY{n}{sympy}\PY{o}{.}\PY{n}{Eq}\PY{p}{(}\PY{n}{sympy}\PY{o}{.}\PY{n}{MatrixSymbol}\PY{p}{(}\PY{l+s}{\PYZdq{}}\PY{l+s}{\PYZbs{}}\PY{l+s}{mathbf\PYZob{}A\PYZcb{}\PYZus{}\PYZob{}PCA\PYZcb{}}\PY{l+s}{\PYZdq{}}\PY{p}{,}\PY{l+m+mi}{3}\PY{p}{,}\PY{l+m+mi}{3}\PY{p}{)}\PY{p}{,}
                 \PY{n}{sympy}\PY{o}{.}\PY{n}{Matrix}\PY{p}{(}\PY{n}{A\PYZus{}pca}\PY{o}{.}\PY{n}{round}\PY{p}{(}\PY{n}{decimals}\PY{o}{=}\PY{l+m+mi}{4}\PY{p}{)}\PY{p}{)}\PY{p}{)}
\end{Verbatim}
\texttt{\color{outcolor}Out[{\color{outcolor}7}]:}
    
    
        \begin{equation*}
        \mathbf{A}_{{PCA}} = \left[\begin{matrix}0.6591 & 0.7432 & 0.1154\\0.1426 & -0.2742 & 0.951\\0.7384 & -0.6104 & -0.2867\end{matrix}\right]
        \end{equation*}

    


    \subsection{Aplicación del método ICA}


    La aplicación del método de componentes independientes utizada es
provista en la biblioteca \emph{scikit-learn} y utiliza el algorímtmo de
maximización de la no-gaussianidad \emph{FastICA} \cite{fastica}. En
concreto, se basa en la maximización de la negentropía a partir de un
esquema iterativo de punto fijo. Internamente, la implementación aplica
PCA en el preprocesamieto para blaquear los datos.

El resultado de la aplicación este método para las imágenes provistas se
muestran el siguientes lineas:

    \begin{Verbatim}[commandchars=\\\{\}]
{\color{incolor}In [{\color{incolor}8}]:} \PY{n}{np}\PY{o}{.}\PY{n}{random}\PY{o}{.}\PY{n}{seed}\PY{p}{(}\PY{l+m+mi}{27}\PY{p}{)}
        
        \PY{n}{ica} \PY{o}{=} \PY{n}{decomposition}\PY{o}{.}\PY{n}{FastICA}\PY{p}{(}\PY{n}{n\PYZus{}components}\PY{o}{=}\PY{l+m+mi}{3}\PY{p}{)}
        \PY{n}{img\PYZus{}ica}\PY{o}{=} \PY{n}{ica}\PY{o}{.}\PY{n}{fit\PYZus{}transform}\PY{p}{(}\PY{n}{Y}\PY{o}{.}\PY{n}{T}\PY{p}{)}
        
        \PY{n}{f}\PY{p}{,} \PY{p}{(}\PY{n}{ax1}\PY{p}{,}\PY{n}{ax2}\PY{p}{,}\PY{n}{ax3}\PY{p}{)} \PY{o}{=} \PY{n}{plt}\PY{o}{.}\PY{n}{subplots}\PY{p}{(}\PY{l+m+mi}{1}\PY{p}{,}\PY{l+m+mi}{3}\PY{p}{)}
        \PY{n}{f}\PY{o}{.}\PY{n}{set\PYZus{}size\PYZus{}inches}\PY{p}{(}\PY{p}{(}\PY{l+m+mi}{24}\PY{p}{,}\PY{l+m+mi}{40}\PY{p}{)}\PY{p}{)}
        \PY{n}{ax1}\PY{o}{.}\PY{n}{imshow}\PY{p}{(}\PY{n}{img\PYZus{}ica}\PY{p}{[}\PY{p}{:}\PY{p}{,}\PY{l+m+mi}{0}\PY{p}{]}\PY{o}{.}\PY{n}{reshape}\PY{p}{(}\PY{p}{(}\PY{l+m+mi}{512}\PY{p}{,}\PY{l+m+mi}{512}\PY{p}{)}\PY{p}{)}\PY{p}{,} \PY{n}{cmap}\PY{o}{=}\PY{n}{plt}\PY{o}{.}\PY{n}{cm}\PY{o}{.}\PY{n}{gray}\PY{p}{)}
        \PY{n}{ax1}\PY{o}{.}\PY{n}{set\PYZus{}title}\PY{p}{(}\PY{l+s}{\PYZdq{}}\PY{l+s}{Componente Independiente 1}\PY{l+s}{\PYZdq{}}\PY{p}{)}
        \PY{n}{ax2}\PY{o}{.}\PY{n}{imshow}\PY{p}{(}\PY{n}{img\PYZus{}ica}\PY{p}{[}\PY{p}{:}\PY{p}{,}\PY{l+m+mi}{1}\PY{p}{]}\PY{o}{.}\PY{n}{reshape}\PY{p}{(}\PY{p}{(}\PY{l+m+mi}{512}\PY{p}{,}\PY{l+m+mi}{512}\PY{p}{)}\PY{p}{)}\PY{p}{,} \PY{n}{cmap}\PY{o}{=}\PY{n}{plt}\PY{o}{.}\PY{n}{cm}\PY{o}{.}\PY{n}{gray}\PY{p}{)}
        \PY{n}{ax2}\PY{o}{.}\PY{n}{set\PYZus{}title}\PY{p}{(}\PY{l+s}{\PYZdq{}}\PY{l+s}{Componente Independiente 2}\PY{l+s}{\PYZdq{}}\PY{p}{)}
        \PY{n}{ax3}\PY{o}{.}\PY{n}{imshow}\PY{p}{(}\PY{n}{img\PYZus{}ica}\PY{p}{[}\PY{p}{:}\PY{p}{,}\PY{l+m+mi}{2}\PY{p}{]}\PY{o}{.}\PY{n}{reshape}\PY{p}{(}\PY{p}{(}\PY{l+m+mi}{512}\PY{p}{,}\PY{l+m+mi}{512}\PY{p}{)}\PY{p}{)}\PY{p}{,} \PY{n}{cmap}\PY{o}{=}\PY{n}{plt}\PY{o}{.}\PY{n}{cm}\PY{o}{.}\PY{n}{gray}\PY{p}{)}
        \PY{n}{ax3}\PY{o}{.}\PY{n}{set\PYZus{}title}\PY{p}{(}\PY{l+s}{\PYZdq{}}\PY{l+s}{Componente Independiente 3}\PY{l+s}{\PYZdq{}}\PY{p}{)}
\end{Verbatim}

            \begin{Verbatim}[commandchars=\\\{\}]
{\color{outcolor}Out[{\color{outcolor}8}]:} <matplotlib.text.Text at 0x109729410>
\end{Verbatim}
        
    \begin{center}
    \adjustimage{max size={0.9\linewidth}{0.9\paperheight}}{SepFuentes_files/SepFuentes_30_1.png}
    \end{center}
    { \hspace*{\fill} \\}
    
    Se aprecia claramente que el método ICA ha permitido separar las
componentes independientes que estaban mezcladas en las imágenes
provistas. Las tres imagenes son perfectamente identificables despues de
la tranformación. Se aprecia una inversión del color segunda componente,
lo que pueden explicarse dado que las componentes por ICA pueden incluir
la multiplicación por un un escalar (negativo como en este caso) o
comutaciones.La matríz de mezcla \(\mathbf{A}_{ICA}\) en este caso
tambien se puede estimar como:

    \begin{Verbatim}[commandchars=\\\{\}]
{\color{incolor}In [{\color{incolor}9}]:} \PY{n}{sympy}\PY{o}{.}\PY{n}{Eq}\PY{p}{(} \PY{n}{sympy}\PY{o}{.}\PY{n}{MatrixSymbol}\PY{p}{(}\PY{l+s}{\PYZdq{}}\PY{l+s}{\PYZbs{}}\PY{l+s}{mathbf\PYZob{}A\PYZcb{}\PYZus{}\PYZob{}ICA\PYZcb{}}\PY{l+s}{\PYZdq{}}\PY{p}{,}\PY{l+m+mi}{3}\PY{p}{,}\PY{l+m+mi}{3}\PY{p}{)}\PY{p}{,}
                 \PY{n}{sympy}\PY{o}{.}\PY{n}{Matrix}\PY{p}{(}\PY{n}{ica}\PY{o}{.}\PY{n}{mixing\PYZus{}}\PY{o}{.}\PY{n}{round}\PY{p}{(}\PY{n}{decimals}\PY{o}{=}\PY{l+m+mi}{4}\PY{p}{)}\PY{p}{)}\PY{p}{)}
\end{Verbatim}
\texttt{\color{outcolor}Out[{\color{outcolor}9}]:}
    
    
        \begin{equation*}
        \mathbf{A}_{{ICA}} = \left[\begin{matrix}-566.9308 & 75.3859 & 675.8539\\-241.3357 & -183.6017 & -37.1631\\-419.1685 & -580.6624 & 630.7346\end{matrix}\right]
        \end{equation*}

    


    \section{Conclusiones}


    Se han utilizado los métodos de análisis por Componentes Principales y
Componentes Independientes para intentar separar tres imágenes mezcladas
linealmente en otras tres imágenes.

Los resultados obtenidos por el método de Componentes Principales han
permitido descubrir tranformar los datos a una nueva base en la que las
nuevas componentes estan descorrelacionadas. El resultado de este método
perimite una mayor distinción entre las imágenes pero se sigue
observando una mezcla parcial.

El método de componentes independientes, sim embargo, se adecúa mucho
mas al problema propuesto y es capaz de reproducir las componentes
originales salvo factores y conmutaciones con exactitud.


    % Add a bibliography block to the postdoc
    
    
\bibliographystyle{unsrt}
\bibliography{ipython}

    
    \end{document}
